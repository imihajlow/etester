\documentclass[a4paper,12pt,twoside]{article}
\usepackage{anysize}
\begin{document}
\pagestyle{empty}
\marginsize{1.5in}{1in}{1in}{1in}
\linespread{1.5}
\begin{flushright}
\textbf{Mikhailov Ivan Victorovich}

Date of birth: 1990

Department of Computer Technologies and Controlling Systems,

Chair of Computer Technology, group 6114

\underline{Direction of preparation:}

230100 Computer Technology and Informatics

e-mail: imihajlow@gmail.com
\end{flushright}

\begin{center}
\textbf{Modular system for automatic testing of electornic equipment}

\textbf{I. V. Mikhailov}

\textbf{Supervisor --- docent, Cand. Tech. Sci. P. V. Kustarev}
\end{center}

On the electronic productions cheap and reliable quality control is needed.
Manual quality control using simple tools like oscilloscope only allows to test small lots of electronic equipment. In addition, it's impossible to perform a manual stress-tests.

The goal of the project was to develop the complete automatic test-bench for electronic equipment.

Following tasks were resolved to achieve this goal:
\begin{itemize}
\item Analysis of modern test description languages;
\item Development of hardware architecture of the testing complex, specification of the interfaces and making up the technical requirements for the hardware modules implementation;
\item Development of the system-on-a-chip of the FPGA-based main module;
\item Making up the technical requirements for the software tools used for loading the test scripts and results visualization.
\end{itemize}

As a result, the automatic hardware testing system was developed and created. The use of the system in the QA lead to costs reduction in post-production testing and technical support.

The developed system is in use by Stardex Oy for the post-production testing.
\end{document}
